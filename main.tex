\documentclass[letterpaper,10pt]{article}
\usepackage[utf8]{inputenc}
\usepackage[sfdefault]{FiraSans} % Font
\usepackage[T1]{fontenc}
\usepackage{geometry}
\geometry{left=1in,right=1in,top=1in,bottom=1in} % Adjust margins as needed
\usepackage{enumitem} % For customizable lists
\setlist[itemize, 1]{leftmargin=0pt} % No indentation for first level of itemize
\setlist[itemize]{label={}} % Empty label for bullet list
\usepackage{fontawesome5} % For social media icons
\usepackage{hyperref} % For clickable links
\usepackage{array} % Required for newcolumntype
\usepackage{tabularx} % Required for tabularx and X column type
\hypersetup{
    colorlinks=true,
    linkcolor=blue,
    filecolor=magenta,
    urlcolor=blue,
}

\newcolumntype{L}{>{\raggedright\arraybackslash}p{\dimexpr0.75\linewidth - 2\tabcolsep}}
\newcolumntype{R}{>{\raggedleft\arraybackslash}X}
\newcommand{\dateitem}[2]{
    \item
    \noindent
    \begin{tabularx}{\linewidth}{@{}LR@{}}
        #1 & #2 \\
    \end{tabularx}
    \par
}

% MS Word style justified text
\tolerance=1
\emergencystretch=\maxdimen
\hyphenpenalty=10000
\hbadness=10000

% Define colors
\usepackage{xcolor}
\definecolor{headerblue}{HTML}{2A5E7D} % Darker blue for headers
\definecolor{textgray}{HTML}{333333} % Dark gray for body text

% Custom section formatting
\usepackage{titlesec}
\titleformat{\section}
  {\normalfont\Large\bfseries\color{headerblue}}% format
  {}% label
  {0pt}% separation
  {}% before code
  [\titlerule]% after code
\titlespacing{\section}{0pt}{12pt}{6pt} % Adjust spacing as needed
\titleformat{\subsection}
  {\normalfont\large\bfseries\color{headerblue}}% format
  {}% label
  {0pt}% separation
  {}% before code
  {}% no rule after for subsection
\titlespacing{\subsection}{0pt}{6pt}{6pt}

% --- BEGIN DOCUMENT ---
\begin{document}

% --- HEADER ---
\begin{center}
    \vspace{-1cm} % Adjust vertical spacing as needed
    \color{headerblue}
    {\Huge \bfseries Rafael C. Giménez} \\
    \vspace{5pt}
    \normalsize
    MSc Artificial Intelligence $|$ Biomedical Engineer \\
    \vspace{5pt}
    \color{textgray}
    \href{mailto:rafael.cga@proton.me}{rafael.cga@proton.me} $|$ Madrid, Spain \\
    \vspace{3pt}
    \href{https://www.linkedin.com/in/rafaelcga/}{\faLinkedin\ LinkedIn} $|$ \href{https://github.com/rafaelcga}{\faGithub\ GitHub} \\
\end{center}
\vspace{5pt} % Space after header

% --- SUMMARY ---
\section*{Summary}
\vspace{5pt}
\color{textgray}
Rafael C. Giménez is an AI and Biomedical Engineer with expertise in Medical Imaging and Computer Vision. He holds an MSc in AI and a BE in Biomedical Engineering. His research experience includes nanoparticle characterization and biomedical image analysis using AI at Universidad Politécnica de Madrid, Natural Language Processing at Hospital 12 de Octubre, and X-Ray Computed Tomography algorithms at Universidad Carlos III de Madrid. He also has experience as an Assistant Professor and has published in both high-impact Medical Imaging journals and Biomedical Engineering congresses.

Eager to transition from academia, he seeks to apply his research foundation and analytical skills to real-world challenges within an enterprise environment, actively contributing to and learning from collaborative teams.

% --- EDUCATION ---
\section*{Education}
\vspace{5pt}
\color{textgray}
\begin{itemize}
    \dateitem{\textbf{PhD in Software, Systems and Computation}}{Sep 2022 -- Present}
        Universidad Politécnica de Madrid, Madrid, Spain
    \dateitem{\textbf{MSc in Artificial Intelligence}}{Sep 2021 -- Jul 2022}
        Universidad Politécnica de Madrid, Madrid, Spain
    \dateitem{\textbf{BE in Biomedical Engineering}, Major in Medical Imaging}{Sep 2017 -- Jul 2021}
        Universidad Carlos III de Madrid, Leganés, Spain
\end{itemize}

% --- EXPERIENCE ---
\section*{Experience}
\vspace{5pt}
\color{textgray}
\subsection*{\textbf{Universidad Politécnica de Madrid}}
\begin{itemize}
    \dateitem{\textbf{Researcher} at the \textbf{Biomedical Informatics Group}}{Jul 2023 -- Present}
        Research on nanoparticle characterization in Electron Microscopy images through Artificial Intelligence.
    \dateitem{\textbf{Assistant Professor}}{Sep 2023 -- Jun 2024}
        Teaching position in Programming I and Programming II within the Computer Science Bachelor's Degree.
    \dateitem{\textbf{Research Collaborator} at the \textbf{Biomedical Informatics Group}}{Jan 2022 -- Jul 2023}
        Research on Computer Vision algorithms applied to biomedical image characterization.
\end{itemize}
\subsection*{\textbf{Hospital 12 de Octubre}}
\begin{itemize}
    \dateitem{\textbf{Researcher}}{Dec 2022 -- Jul 2023}
        Research on Natural Language Processing for phenotypic ontology translation and information extraction.
\end{itemize}
\subsection*{\textbf{Universidad Carlos III de Madrid}}
\begin{itemize}
    \dateitem{\textbf{Research Intern} at the \textbf{Biomedical Imaging and Instrumentation Group}}{Jul 2020 -- Jun 2021}
        Development of advanced callibration and reconstruction algorithms for X-Ray Computed Tomography using Artificial Intelligence.
\end{itemize}

% --- PUBLICATIONS ---
\section*{Publications}
\vspace{5pt}
\color{textgray}
\begin{itemize}
    \dateitem{\textbf{Multi-Class Dental CBCT Segmentation in Data-Constrained Scenarios Through Transformers}}{Mar 2025}
        \textcolor{darkgray}{\textbf{Giménez-Aguilar, R. C.}}, Paraíso-Medina, S., García-Remesal, M., Pradíes- Ramiro, G. J., Bonfanti-Gris, M., \& Alonso-Calvo, R.
        \\ \textit{International Journal of Interactive Multimedia and Artificial Intelligence}
        \\ \href{https://dx.doi.org/10.9781/ijimai.2025.03.003}{10.9781/ijimai.2025.03.003}
    \dateitem{\textbf{Deep Learning–Based Estimation of Radiographic Position to Automatically Set Up the X-Ray Prime Factors}}{Oct 2024}
        Del Cerro, C. F., \textcolor{darkgray}{\textbf{Giménez, R. C.}}, García-Blas, J., Sosenko, K., Ortega, J. M., Desco, M., \& Abella, M.
        \\ \textit{Journal of Imaging Informatics in Medicine}
        \\ \href{https://doi.org/10.1007/s10278-024-01256-x}{10.1007/s10278-024-01256-x}
\end{itemize}

% --- CONGRESSES ---
\section*{Congresses}
\vspace{5pt}
\color{textgray}
\begin{itemize}
    \dateitem{\textbf{Reconocimiento Automático de Posiciones Radiográficas en Radiología}}{Nov 2021}
        \textcolor{darkgray}{\textbf{Giménez, R. C.}}, Del Cerro, C. F., Sosenko, K., Desco, M., García-Blas, J., \& Abella, M.
        \\ \textit{CASEIB 2021: XXXIX Congreso Anual de La Sociedad Española de Ingeniería Biomédica}
    \dateitem{\textbf{Estimación del desplazamiento horizontal del detector en un sistema de rayos X utilizando aprendizaje por transferencia.}}{Nov 2020}
        Del Cerro, C. F., \textcolor{darkgray}{\textbf{Giménez, R. C.}}, Olmos, P. M., Piol, A., Desco, M., \& Abella, M.
        \\ \textit{CASEIB 2020: XXXVIII Congreso Anual de La Sociedad Española de Ingeniería Biomédica}
    \dateitem{\textbf{Corrección del Artefacto de Truncamiento en TAC mediante Aprendizaje profundo}}{Nov 2020}
        Berdón, P. M., Del Cerro, C. F., \textcolor{darkgray}{\textbf{Giménez, R. C.}}, Desco, M., \& Abella, M.
        \\ \textit{CASEIB 2020: XXXVIII Congreso Anual de La Sociedad Española de Ingeniería Biomédica}
\end{itemize}

% --- AWARDS & CERTIFICATIONS ---
\section*{Awards \& Certifications}
\vspace{5pt}
\color{textgray}
\begin{itemize}
    \dateitem{\textbf{Madrid Community Excellence Scholarship (2020/2021)}}{Jul 2021}
        Consejería de Ciencia, Universidades e Innovación de la Comunidad de Madrid
    \dateitem{\textbf{IELTS Academic}, 8.0/9.0}{Dec 2020}
        Cambridge English
    \dateitem{\textbf{Collaboration Scholarship at the Department of Bioengineering and Aerospace Engineering (UC3M)}}{Nov 2020}
        Ministerio de Educación y Formación Profesional
    \dateitem{\textbf{Madrid Community Excellence Scholarship (2019/2020)}}{Jan 2020}
        Consejería de Ciencia, Universidades e Innovación de la Comunidad de Madrid
    \dateitem{\textbf{First Certificate in English}, Band A (C1)}{Dec 2015}
        Cambridge English
\end{itemize}

% --- SKILLS ---
\section*{Skills}
\vspace{5pt}
\color{textgray}
\begin{itemize}
    \item \textbf{Programming Languages:} Python, MATLAB, Java (basic), Bash (basic)
    \item \textbf{Tools/Technologies:} Git, Docker, Podman, Linux
\end{itemize}

% --- LANGUAGES ---
\section*{Languages}
\vspace{5pt}
\color{textgray}
\begin{itemize}
    \item \textbf{Spanish:} Native
    \item \textbf{English:} Professional level
\end{itemize}


\end{document}
